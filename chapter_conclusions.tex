\url{https://ptpb.pw/b7_V.png}

This thesis presents a collection of static and dynamic snapshots of diverse
microbial systems. Although our new methods are in no way specific to the human
gut, part of our focus has been to bridge the gap between computation and
biology, to get us closer towards a future where drug prescription, diet and
recreational activities can be further \textit{personalized} -- through a layer
invisible to the naked eye but perceivable in most respects of life.

It seems impossible to accurately predict the impact of scientific discoveries, 
historically, we have seen great evidence of seemingly anachronistic findings 
(for example neural networks \cite{Tem10}) that later, usually when other 
technologies catch up, become new fields of research or cornerstones in 
consumer applications.  Therefore, while acknowledging this is a tough problem, 
I present my personal view on future directions that can build up from the 
contributions in this thesis. I divide these into three sections: diagnostic 
methods, treatment and analysis.

\section{Diagnostic Methods}

In Chapter~\ref{chapter_dogs} and Chapter~\ref{chapter_ibds}, we presented
examples of biomarkers for \gls{ibd} and \gls{cd} from two different analytical 
perspectives.  First, in a dog cohort, we showed that we can reformulate the 
dysbiosis index we originally developed for humans \cite{Gevers2014}, and make 
it specific for a different host\hyp{}species. Both in humans and dogs, this 
log-ratio of bacterial groups is associated with decreased diversity, and in 
humans, it is also associated with inflammation. Next, after noticing the 
increased volatility in the microbiome of subjects with \gls{ibd}, we 
benchmarked microbial variability in fecal samples as an effective classifier 
for disease.  This method is powerful, because it overcomes the low 
classification accuracy of fecal samples, and improves as more samples per 
subject are collected.

Although both approaches are encouraging, translating research into 
consumer\hyp{}level applications, commonly presents formidable challenges that 
might only be solvable by future generations.  Take for example the \gls{ecg}, 
it took almost 125 years, between the first observation of 
biopotentials\footnote{Credited to Luigi Galvani in 1787.}, and the moment when 
the first table \gls{ecg}\footnote{Willem Einthoven's ectrocardiograph was 
manufactured by the Cambridge Scientific Instrument Company of London in 1911.} 
became commercially available \cite{ECGZywietz}. Even after the tremendous 
progress, proper validation of computer-generated diagnostics only appeared 
more than 200 years after the initial discovery \cite{njem_ecg}.

I use the example of the \gls{ecg}, because much like raw heart biopotentials,
(\textit{i}) microbiome data is plagued with noise, and (\textit{ii})
determining the appropriate filters and thresholds directly depends on the
use-case. However, unlike with the early developments of the \gls{ecg}, we live
in digital and connected world. As such, the following focuses will
shorten the time between future discoveries and innovation:

\begin{description}
    \item[Mechanistic Studies]The novelty characteristic of microbiome studies, 
        has produced a large amount of descriptive studies, contrastingly, 
        mechanistic experiments have lagged in validating several of these 
        findings (likely due to the more expensive requirements).  If the goal 
        is to relate the presence or abundance of a microbiome feature to 
        disease state or biochemical process, the underlying methods must be 
        informed by biological inferences in order for these biomarkers to be 
        credible.

    \item[Open Data]Medical research is specially affected by human factors, 
        being able to consistently collect samples from a subject is not always 
        easy or deterministic (think of bowel problems and fecal samples). A 
        powerful practice used to counter underpowered studies, is to improve 
        on the current sample size through the reuse of previously published 
        data. The availability of open resources like 
        Qiita\footnote{\url{https://qiita.ucsd.edu/}}, makes data reuse 
        seamless. Importantly, the work in this thesis was only possible 
        through the reuse of existing datasets (see 
        Chapter~\ref{exploratory_chapter} through Chapter~\ref{chapter_fmts}).  
        Although making data openly available is an important first step, 
        proper standardization of processing protocols, will also be key to 
        maximizes data re-usability, a remarkable example of this practice is 
        the \gls{emp} \cite{RN4267}.
\end{description}

\section{Treatment}

Microbiome\hyp{}based treatments, are generally based on the transplantation of 
microbial communities from a \textit{healthy} donor into an affected 
\textit{recipient}. Three spearheading examples are: \glspl{fmt} to treat 
\gls{cdi} \cite{RN4129}, skin microbiomes transplants to treat atopic 
dermatitis \cite{GalloSkin}, and (although still in early stages) a capsule 
full of microbial spores to treat \gls{uc} by Seres Therapeutics.

Although most \glspl{fmt} succeed at treating \gls{cdi}, what makes a
successful transplant is not clear yet, the number of variables implicated in
answering this question is immense. From a computational complexity standpoint,
\textit{a priori} determination of whether or not a new community can colonize 
an existing ecosystem is considered a hard problem, belonging to the 
\textbf{\#-P} class \cite{RN4266}. As such our focus should be on strengthening 
our systematic understanding of the transplant, and characterize not only what 
makes a successful transplant but also what makes a failed one.  For example, 
the impact of the work presented in Section~\ref{section_fmt} could have been 
magnified, if we had included subjects for which the \gls{fmt} failed. With 
appropriate sample sizes, we could have applied a number of techniques to 
single out (if any) the features common that lead to a failed transplant. With 
this knowledge, we could pre-screen \textit{donors} and \textit{recipients} to 
ensure a successful transplant.

\section{Analysis}

Much like our microbial symbionts depend on the host environment (and vice
versa), the development and funding of analytical tools depends on the
ever-growing necessity to unravel complex patterns in a number of experimental
setups (cross-sectional, longitudinal, multi-'omic, etc). Thus, these tools,
must take into consideration the ability to be flexible, scalable and when
possible interactive.

Novel analytical methods and software infrastructure, should be built making 
scalability a priority.  We have seen a steady increase in our capability to 
generate data, and it is likely that this trend will continue. Emperor 
(Chapter~\ref{chapter_emperor}) was partly a response to the limitations in 
existing software, and more recently we had to re-architect the underlying 
implementation to cope with modern and larger datasets.

Flexibility and compliance with community standards, makes software available
to a wider audience. Take the count table, often acting as the core
data-structure for metabolomic, transcriptomic, proteomic, (and other 'omics)
analyses. If the software producing this data complies with a standard, like
the \gls{biom}-format, the end user is now free to select from a variety of
methods as opposed to being limited to nieche-software. This idea is being
taken a step further with \gls{qiime} 2, where a semantic type system defines
the methods and visualizations that can be applied to a given dataset.

Finally, the value of interactively exploring a dataset, lies on our ability to
quickly test hypotheses, and iteratively develop new ones. Future exploratory
analysis tools should be developed with interoperability in mind, for example,
brushing to select a group of samples in a view of data might act as a filter
for a different representation. Modern web technologies, and software
development frameworks like \gls{qiime}2 will likely be the pioneers of these
global overviews of microbial diversity.
