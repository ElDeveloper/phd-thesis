%
%
% UCSD Doctoral Dissertation Template
% -----------------------------------
% http:\\ucsd-thesis.googlecode.com
%
%


%% REQUIRED FIELDS -- Replace with the values appropriate to you

% No symbols, formulas, superscripts, or Greek letters are allowed
% in your title.
\title{Statistical Representations Of Microbial Systems}

\author{Yoshiki V\'azquez Baeza}
\degreeyear{2017}

% Master's Degree theses will NOT be formatted properly with this file.
\degreetitle{Doctor of Philosophy} 

\field{Computer Science}
\chair{Professor Rob Knight}
% Uncomment the next line iff you have a Co-Chair
% \cochair{Professor Cochair Semimaster} 
%
% Or, uncomment the next line iff you have two equal Co-Chairs.
%\cochairs{Professor Chair Masterish}{Professor Chair Masterish}

%  The rest of the committee members  must be alphabetized by last name.
\othermembers{
Professor Pieter Dorrestein\\
Professoe  Rachel Dutton\\
Professor  Jim Hollan\\
Professor Larry Smarr\\
}
\numberofmembers{5} % |chair| + |cochair| + |othermembers|


%% START THE FRONTMATTER
%
\begin{frontmatter}

%% TITLE PAGES
%
%  This command generates the title, copyright, and signature pages.
%
\makefrontmatter 

%% DEDICATION
%
%  You have three choices here:
%    1. Use the ``dedication'' environment. 
%       Put in the text you want, and everything will be formated for 
%       you. You'll get a perfectly respectable dedication page.
%   
%
%    2. Use the ``mydedication'' environment.  If you don't like the
%       formatting of option 1, use this environment and format things
%       however you wish.
%
%    3. If you don't want a dedication, it's not required.
%
%
\begin{dedication} 
	To my parents, my family and my friends.
\end{dedication}

% You are responsible for formatting here.
%\begin{mydedication} 
%  \vspace{1in}
%  \begin{flushleft}
%    To me.
%  \end{flushleft}
%   
%   \vspace{2in}
%   \begin{center}
%     And you.
%   \end{center}
%
%  \vspace{2in}
%  \begin{flushright}
%    Which equals us.
%  \end{flushright}
%\end{mydedication}



%% EPIGRAPH
%
%  The same choices that applied to the dedication apply here.
%

% The style file will position the text for you.
\begin{epigraph} 
  \emph{The purpose of computing is insight, not numbers}\\
  ---Richard Hamming
\end{epigraph}

%% SETUP THE TABLE OF CONTENTS
%
\tableofcontents

%%
%% This block was needed to re-format the title of the glossary to match the
%% headings of the list of figures and list of tables.
%%
%% start hack:
\renewcommand{\glossarysection}[2][]{
\newpage
\noindent
\centerline{LIST OF ABBREVIATIONS}
}
%% end hack

\printglossary[title=List of Abbreviations,toctitle=List of Abbreviations,nonumberlist ]
\listoffigures  % Uncomment if you have any figures
\listoftables   % Uncomment if you have any tables

%% ACKNOWLEDGEMENTS
%
%  While technically optional, you probably have someone to thank.
%  Also, a paragraph acknowledging all coauthors and publishers (if
%  you have any) is required in the acknowledgements page and as the
%  last paragraph of text at the end of each respective chapter. See
%  the OGS Formatting Manual for more information.
%
\begin{acknowledgements} 
 Thanks to whoever deserves credit for Blacks Beach, Porters Pub, and
 every coffee shop in San Diego. 

 Thanks also to hottubs.

    Section~\ref{section_review}, in full, is a reprint of the material as it 
    appears in ``Impacts of the human gut microbiome in therapeutics''. Y.  
    V\'azquez-Baeza, C. Callewaert, J. Debelius, E. Hyde, C.  Marotz, J. T.  
    Morton, A. Swafford, A. Vrbanac, P. C.  Dorrestein and R.  Knight.  
    \emph{Annual Review of Pharmacology and Toxicology, 2017}. 58, 2017.

    Section~\ref{section_emperor}, in full, is a reprint of the material as it 
    appears in ``Emperor: a tool for visualizing high throughput microbial 
    community data''. Y. V\'azquez-Baeza, A. Gonzalez, M. Pirrung, R.  Knight.  
    \emph{GigaScience}, 2, 2013 The dissertation/thesis author was the primary 
    investigator and author of this paper. In addition, we thank Jackson Chen, 
    Jai Ram Rideout, Daniel McDonald, William Van Treuren, Jose Antonio 
    Navas\hyp{}Molina, Nickolas A. Bokulich, Adam Robbins\hyp{}Pianka and Greg 
    Caporaso for feedback and useful discussion regarding the design and 
    implementation of the software package. This work was supported in part by 
    the National Institutes of Health, the Crohns and Colitis Foundation of 
    America, the Alfred P. Sloan Foundation, and the Howard Hughes Medical 
    Institute. The dissertation/thesis author was the primary investigator and 
    author of this paper.

    Section~\ref{section_animations}, in full, is a reprint of the material as 
    it appears in ``Bringing the Dynamic Microbiome to Life with Animations''.  
    Y.  V\'azquez-Baeza, A. Gonzalez, L. Smarr, D.  McDonald, J.  T. Morton, J.  
    A.  Navas-Molina, R. Knight. \emph{Cell Host and Microbe}, 11, 2017. The 
    dissertation/thesis author was the primary investigator and author of this 
    paper.

    Section~\ref{dogs}, in full, is a reprint of the material as it appears in 
    ``Dog and human inflammatory bowel disease rely on overlapping yet distinct 
    dysbiosis networks''. Y. V\'azquez-Baeza, E. R. Hyde, J. S.  Suchodolski, 
    R. Knight.  \emph{Nature Microbiology}, 1, 2016. The dissertation/thesis 
    author was the primary investigator and author of this paper. We wish to 
    acknowledge the support provided by the Crohn's and Colitis Foundation of 
    American; the Templeton Foundation and the Keck Foundation (via the Earth 
    Microbiome Project), the National institutes of Health. We wish to thank 
    Zhenjiang Xu, Jon Sanders, Amnon Amir, Gail Ackermann, Jamie Morton, Luke 
    Ursell, Jessica Metcalf, Antonio Gonzalez and Emma Schwager for their 
    useful comments and feedback in the writing of this manuscript. 

    Section~\ref{section_plane}, in full, is a reprint of the material as it 
    appears in ``Dynamics of the human gut microbiome in inflammatory bowel 
    disease''.  J. Halfvarson, C. J. Brislawn, R. Lamendella, Y.  
    V\'azquez-Baeza, W. A. Walters, L. M. Bramer, M. D'Amato, F.  Bonfiglio, D.  
    McDonald, A. Gonzalez, E. E. McClure, M. F. Dunklebarger, R. Knight, J.  K.  
    Jansson. \emph{Nature Microbiology}, 2, 2017. The dissertation/thesis 
    author was the primary analyst and a co-author of this paper.

    Section~\ref{section_ibd}, in full, is a reprint of the material as it 
    appears in ``Guiding longitudinal sampling in inflammatory bowel diseases 
    cohorts''. Y. V\'azquez-Baeza, A. Gonzalez, Z. Zech Xu, A. Washburne, H.  
    Herfarth, R.  B.  Sartor, R. Knight. \emph{Gut}. 2017. The 
    dissertation/thesis author was the primary investigator and author of this 
    paper. We acknowledge grant support by NIH (P01-DK094779), the Crohn's and 
    Colitis Foundation, and the research coordinators of the UNC 
    Multidisciplinary IBD Center at UNC for patient recruitment. Additionally, 
    we thank Gail Ackermann, Jamie Morton and Justin Silverman for their 
    invaluable feedback and discussion provided in the preparation of this 
    manuscript.

    Section~\ref{section_moviefmt}, in full, is a reprint of the material as it 
    appears in ``Dynamic changes in short- and long-term bacterial composition 
    following fecal microbiota transplantation for recurrent Clostridium 
    difficile infection''.  A. Weingarden, A. Gonzalez, Y.  V\'azquez-Baeza, S.  
    Weiss, G.  Humphry, D. Berg-Lyons, D. Knights, T.  Unno, A. Bobr, J.  Kang, 
    A. Khoruts, R. Knight, M. J. Sadowsky. \emph{Microbiome}. 3, 2015.  The 
    dissertation/thesis author was a co-primary investigator and author of this 
    paper. This research was supported, in part, by NIH Grant R21AI091907 (to 
    AK and MJS). ARW was supported by a Doctoral Dissertation Fellowship from 
    the University of Minnesota Graduate School and by the Dennis W.  Watson 
    Fellowship in Microbiology. Work done in the Knight lab was funded by the 
    NIH, Crohn's \& Colitis Foundation of America, and the HHMI.  

    Section~\ref{section_fmt}, in full, is a reprint of the material as it 
    appears in ``Changes in microbial ecology after fecal microbiota 
    transplantation for recurrent C. difficile infection affected by underlying 
    inflammatory bowel disease''. S. Khanna, Y.  V\'azquez-Baeza, A.  Gonzalez, 
    S. Weiss, B.  Schmidt, D. A.  Muñiz-Pedrogo, J. F. Rainey 3rd, P. Kammer, 
    H. Nelson, M.  Sadowsky, A.  Khoruts, S. L. Farrugia, R. Knight, D. S.  
    Pardi, P. C.  Kashyap. \emph{Microbiome}. 5, 2017. The dissertation/thesis 
    author was the co-primary investigator and author of this paper.

\end{acknowledgements}


%% VITA
%
%  A brief vita is required in a doctoral thesis. See the OGS
%  Formatting Manual for more information.
%
\begin{vitapage}
\begin{vita}
  \item[2012] B.~S. in Biomedical Engineering, Universidad Iberoamericana, Ciudad de M\'exico
  \item[2016] M.~Sc. in Computer Science, University of California, San Diego
  \item[2017] Ph.~D. in Computer Science, University of California, San Diego 
\end{vita}


\begin{publications}

    \item \textsl{Author names marked with $\dagger$ indicate shared first co-authorship}.

    \item \textbf{Y. V\'azquez-Baeza}, A. Gonzalez, M. Pirrung, R. Knight. ``Emperor: a tool for visualizing high throughput microbial community data'', \emph{GigaScience}, 2, 2013.

    \item \textbf{Y. V\'azquez-Baeza}, A. Gonzalez, L. Smarr, D. McDonald, J. T. Morton, J. A. Navas-Molina, R. Knight. ``Bringing the Dynamic Microbiome to Life with Animations'', \emph{Cell Host and Microbe}, 11, 2017.

    \item \textbf{Y. V\'azquez-Baeza}, E. R. Hyde, J. S. Suchodolski, R. Knight. ``Dog and human inflammatory bowel disease rely on overlapping yet distinct dysbiosis networks'', \emph{Nature Microbiology}, 1, 2016.

    \item  $\dagger$S. Khanna, \textbf{$\dagger$Y. V\'azquez-Baeza}, A.  
        Gonzalez, S. Weiss, B. Schmidt, D. A. Muñiz-Pedrogo, J. F. Rainey 3rd, 
        P. Kammer, H. Nelson, M. Sadowsky, A. Khoruts, S. L. Farrugia, R.  
        Knight, D. S. Pardi, P. C. Kashyap. ``Changes in microbial ecology 
        after fecal microbiota transplantation for recurrent C. difficile 
        infection affected by underlying inflammatory bowel disease'', 
        \emph{Microbiome}. 5, 2017.

    \item J. Halfvarson, C. J. Brislawn, R. Lamendella, \textbf{Y. V\'azquez-Baeza}, W. A. Walters, L. M. Bramer, M. D'Amato, F. Bonfiglio, D. McDonald, A. Gonzalez, E. E. McClure, M. F. Dunklebarger, R. Knight, J. K. Jansson. ``Dynamics of the human gut microbiome in inflammatory bowel disease'', \emph{Nature Microbiology}, 2, 2017.
  
    \item $\dagger$A. Weingarden, A. Gonzalez, \textbf{$\dagger$Y.  
        V\'azquez-Baeza}, S. Weiss, G. Humphry, D. Berg-Lyons, D. Knights, T.  
        Unno, A. Bobr, J. Kang, A. Khoruts, R. Knight, M. J. Sadowsky.  
        ``Dynamic changes in short- and long-term bacterial composition 
        following fecal microbiota transplantation for recurrent Clostridium 
        difficile infection''. \emph{Microbiome}. 3, 2015.

    \item \textbf{Y. V\'azquez-Baeza}, C. Callewaert, J. Debelius, E. Hyde, C. Marotz, J. T. Morton, A. Swafford, A. Vrbanac, P. C. Dorrestein and R. Knight. ``Impacts of the human gut microbiome in therapeutics''. \emph{Annual Reviews}. 58, 2017.

    \item \textbf{Y. V\'azquez-Baeza}, A. Gonzalez, Z. Zech Xu, A. Washburne, 
        H. Herfarth, R. B. Sartor, R. Knight. ``Guiding longitudinal sampling 
        in inflammatory bowel diseases cohorts''. \emph{Gut}. 2017.

    \item \noindent\rule[0.5ex]{\linewidth}{0.5pt}

    \textsl{The following publications were not included as part of this dissertation, but were also significant byproducts of my doctoral training.}

    \item C. A. Lozupone, J. Stombaugh, A. Gonzalez, G. Ackermann, D. Wendel, \textbf{Y. V\'azquez-Baeza}, J. K. Jansson, J. I. Gordon, R. Knight. ``Meta-analyses of studies of the human microbiota''. \emph{Genome Research}. 23, 2013.

    \item J. A. Navas-Molina, J. M. Peralta-S\'anchez, A. Gonzalez, P. J.  
        McMurdie, \textbf{Y. V\'azquez-Baeza}, Z. Xu, L. K Ursell, C. Lauber, 
        H. Zhou, S. Jin Song, J. Huntley, G. L Ackermann, D. Berg-Lyons, S.  
        Holmes,. Gregory Caporaso, R. Knight. ``Advancing our understanding of 
        the human microbiome using QIIME''. \emph{Methods in enzymology}. 531, 
        2013.

    \item G. E Flores, G. Caporaso, J. B Henley, J. R. Rideout, D. Domogala, J. Chase, J. W Leff, \textbf{Y. V\'azquez-Baeza}, A. Gonzalez, R. Knight, R. R. Dunn, N. Fierer. ``Temporal variability is a personalized feature of the human microbiome''. \emph{Genome biology}. 15, 2014.

    \item S. Lax, D. P Smith, J. Hampton-Marcell, S. M Owens, K. M Handley, N. M Scott, S. M Gibbons, P. Larsen, B. D Shogan, S. Weiss, J. L Metcalf, L. K Ursell, \textbf{Y. V\'azquez-Baeza}, W. Van Treuren, N. A Hasan, M. K Gibson, R. Colwell, G. Dantas, R. Knight, J. A Gilbert. ``Longitudinal analysis of microbial interaction between humans and the indoor environment''. \emph{Science}. 345, 2014.

    \item D. Gevers, S. Kugathasan, L. A Denson, \textbf{Y. V\'azquez-Baeza}, 
        W. Van Treuren, B. Ren, E. Schwager, D. Knights, S. Jin Song, M.  
        Yassour, X. C Morgan, A. D Kostic, C. Luo, A. Gonzalez, D. McDonald, Y.  
        Haberman, T. Walters, S. Baker, J. Rosh, M. Stephens, M. Heyman, J.  
        Markowitz, R. Baldassano, A. Griffiths, F. Sylvester, D. Mack, S. Kim, 
        W. Crandall, J. Hyams, C. Huttenhower, R. Knight, R. J Xavier. ``The 
        treatment-naive microbiome in new-onset Crohn's disease''. \emph{Cell 
        Host \& Microbe}. 3, 2014.

    \item $\dagger$A. Gonzalez, \textbf{$\dagger$Y. V\'azquez-Baeza}, R.  
        Knight. ``SnapShot: the human microbiome''. \emph{Cell}. 158, 2014.

    \item $\dagger$A. Gonzalez, \textbf{$\dagger$Y. V\'azquez-Baeza},  J. B. Pettengill, A. Ottesen, D. McDonald, R. Knight. ``Avoiding Pandemic Fears in the Subway and Conquering the Platypus''. \emph{mSystems}. 1, 2016.

    \item R. A Quinn, J. A Navas-Molina, E. R Hyde, S. Jin Song, \textbf{Y. V\'azquez-Baeza}, G. Humphrey, J. Gaffney, J. J Minich, A. V Melnik, J. Herschend, J. DeReus, A. Durant, R. J Dutton, M. Khosroheidari, C. Green, R. da Silva, P. C Dorrestein, R. Knight ``From sample to Multi-Omics conclusions in under 48 Hours''. \emph{mSystems}. 1, 2016.

    \item J. W Debelius, \textbf{Y. V\'azquez-Baeza}, D. McDonald, Z. Xu, E. Wolfe, R. Knight. ``Turning participatory microbiome research into usable data: lessons from the American Gut Project''. \emph{Journal of microbiology \& biology education}. 1, 2016.

    \item J. Debelius, S. Jin Song, \textbf{Y. V\'azquez-Baeza}, Z. Zech Xu, A. Gonzalez, R. Knight. ``Tiny microbes, enormous impacts: what matters in gut microbiome studies?''. \emph{Genome Biology}. 17, 2017.

    \item J. T Morton, J. Sanders, R. A Quinn, D. McDonald, A. Gonzalez, \textbf{Y. V\'azquez-Baeza}, J. A Navas-Molina, S. Jin Song, J. L Metcalf, E. R Hyde, M. Lladser, P. C Dorrestein, R. Knight. ``Balance trees reveal microbial niche differentiation''. \emph{mSystems}. 2, 2017.

    \item S. Weiss, Z. Zech Xu, S. Peddada, A. Amir, K. Bittinger, A. Gonzalez, C. Lozupone, J. R. Zaneveld, \textbf{Y. V\'azquez-Baeza}, A. Birmingham, E. R Hyde, R. Knight ``Normalization and microbial differential abundance strategies depend upon data characteristics''. \emph{Microbiome}. 5, 2017.

    \item L. R. Thompson, J. G. Sanders, D. McDonald, A. Amir, J. Ladau, K. J.  
        Locey, R. J. Prill, A. Tripathi, S. M.  Gibbons, G. Ackermann, J. A.  
        Navas-Molina, S. Janssen, E. Kopylova, \textbf{Y. V\'azquez-Baeza}, A.  
        Gonzalez, J. T. Morton, S. Mirarab, Z. Z. Xu, L. Jiang, M. F.  Haroon, 
        J.  Kanbar, Q.  Zhu, S. Song, T. Kosciolek, N. A. Bokulich J. Lefler, 
        C. J.  Brislawn, G. Humphrey, S. M. Owens, J. Hampton-Marcell, D.  
        Berg-Lyons, V. McKenzie, N. Fierer, J. A. Fuhrman, A. Clauset, R. L.  
        Stevens, A.  Shade, K. S. Pollard, K. D. Goodwin, J. K. Jansson, J. A.  
        Gilbert, R.  Knight and The Earth Microbiome Project Consortium. ``A 
        communal catalogue reveals Earth’s multiscale microbial diversity''.
        \emph{Nature}. 2017.
\end{publications}

\end{vitapage}


%% ABSTRACT
%
%  Doctoral dissertation abstracts should not exceed 350 words. 
%   The abstract may continue to a second page if necessary.
%
\begin{abstract}

    Technological developments in the past thirty years, have transformed 
    sequencing\hyp{}based microbiology into a data-intensive field where 
    computing, and efficient representations are catalyzers of insight into 
    omnipresent and complex microbial interactions. Notably, classical 
    ecologists have set the foundations for the way we analyze these systems, 
    with some techniques dating back to the beginning of the twentieth century.  
    In this thesis, we expand and where possible reuse these techniques, to 
    unravel the hidden patterns comprising the human gut microbiome.

    To set an appropriate motivation and context to the rest of this work, 
    Chapter 1 reviews recent discoveries on the human microbiome and how 
    the communities within can be determinants on the action of therapeutic 
    agents. Next, in Chapter 2, we introduce EMPeror, an interactive analysis 
    and visualization tool that is crucial to the findings presented in later 
    chapters.

    The following three chapters study concrete examples where the microbiome
    has been implicated as a driver or marker for dysbiosis. Chapter 3
    describes how the microbial signature associated with \gls{cd} in humans,
    described in our previous work \cite{RN154}, is overlapping but distinct to
    that of dogs affected with \gls{ibd}. Surprisingly, unlike with humans, dog
    fecal samples alone are strong indicators of the disease. In Chapter 4, we
    study \gls{ibd} from a longitudinal perspective, revealing increased
    volatility in the gut microbiomes of subjects with \gls{ibd}, a property
    that does not appear to be present in unaffected controls. Furthermore, we
    use this as a predicting feature of the disease, and improve on the
    classification accuracy possible through a single fecal sample. In Chapter
    5, we study the effect of \glspl{fmt} to treat \glspl{cdi} and using the
    techniques described in Chapter 2, we show the first animated visualization
    of this process, a dramatic microbial transformation as the subjects
    recover from all \gls{cdi} symptoms. In addition, for \gls{cdi} patients
    who also suffer from a subtype of \gls{ibd}, a treatment with a \gls{fmt}
    results in an increased number of relapses and decreased microbial
    diversity.

    The closing chapter discusses these results and their possible
    applications, as well as future directions for computationally-centric
    microbiome research.

\end{abstract}


\end{frontmatter}
