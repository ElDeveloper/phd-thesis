\chapter{Sample collection and processing for 
Chapter~\ref{chapter_dogs}}\label{appendix_dogs}

\section{Methods}

Naturally passed fecal samples were analyzed from 85 healthy dogs and 65 dogs 
with chronic signs of gastrointestinal disease and confirmed inflammatory 
changes on histopathology. All dogs participated in different clinical studies 
and leftover fecal samples were utilized for this study. The protocol for 
sample collection was approved by the Clinical Research Review Committee of the 
College of Veterinary Medicine, Texas A\&M University (CRRC\#09-06). 

Dogs with clinical signs of chronic \gls{gi} disease (i.e., vomiting, diarrhea, 
anorexia, weight loss, etc.) were diagnosed with idiopathic \gls{ibd} based on 
the\gls{wsava} criteria: (i) chronic (i.e., $>$ 3 weeks) GI signs; (ii) 
histopathologic evidence of mucosal inflammation; (iii) inability to document 
other causes of GI inflammation; (iv) inadequate response to dietary, 
antibiotic, and anthelmintic therapies, and (v) clinical response to 
anti-inflammatory or immunosuppressive agents.  Histological samples were 
obtained endoscopically. The clinical status of each dog was evaluated using a 
published clinical \gls{cibdai}. Within the \gls{ibd} dogs, 41 dogs had 
histological confirmed inflammation in the small intestine, 18 dogs had 
histological changes in both small intestine and colon, and 5 dogs had only 
histological changes reported in the colon. Histological changes were 
predominantly of lymphoplasmacytic infiltrates, with a subset of dogs also 
showing eosinophilic and/or neutrophilic components. The mean (SD) \gls{cibdai} 
for \gls{ibd} dogs was 6.4 (3.1).

Dogs were excluded if they received antibiotics within the past 2 weeks of 
sample collection. Data on antibiotic history was nevertheless collected: 34/65 
dogs with \gls{ibd} had no history of prior antibiotics administration, while 
13 dogs received antibiotics several weeks ($>$2) or months before sample 
collection. The remaining 18 dogs in the \gls{ibd} group had no information 
about prior antibiotic use. In the healthy group (n=85), 76 dogs had not 
received any antibiotics, and 9 dogs had a history of antibiotic use, but not 
within the last 2 weeks of sample collection. No technical replicates were 
generated in this study.

Sample and animal information (i.e., age, weight, gender, breed, duration of 
clinical signs, histopathology, antibiotic usage) was obtained from clinical 
records. Also, if the owner provided the information, the exact diet (trade 
name and manufacturer) fed at time of sample collection was recorded in the 
clinical records, and the dietary macronutrients (protein, fat, and 
carbohydrate content) were recorded from manufacturer's provided data on the 
labels.

Body weights ranged from 2.9 to 55 kg (mean 22 kg, SD: 14.9 kg), which was not 
significantly different from (Mann Whitney test; p=0.087) the healthy dogs 
(range 0.9 to 50 kg; mean 20.3 kg, SD: 10.7g). Mean age (SD) was 5.4 (3.07) in 
the \gls{ibd} group, which was not significantly (Mann Whitney test; p=0.311) 
different from healthy dogs (4.7, 3.2). There was a wide breed distribution 
with 37 different breeds in the \gls{ibd} group and 42 different breeds in the 
control group. In the \gls{ibd} group, Yorkshire terrier, German Shepherd dogs 
and Labrador Retrievers (n=5 each) were most commonly represented.

\Gls{bcs} were assessed according to the \gls{wsava} criteria. \Gls{bcs} is 
rated in a 9-point scale that visually evaluates a dog's body composition. This 
score has been validated against the standard \gls{dexa} \cite{RN4000}. For 
this dataset, the \Gls{bcs} was restricted to a subset of the healthy samples, 
therefore \gls{ibd} vs. Healthy comparisons could not be made in this case.

\subsection{DNA Extraction and Sequencing}

DNA isolation was performed as described by the \gls{emp} Protocol (version 
4\textunderscore 13) for 16S rRNA\cite{RN164}. The full cohort included 192 
samples, of which 15 were removed because those subjects had acute hemorrhagic 
diarrhea, and had little clinical information available.  The remaining 28 
samples did not recover enough sequences after quality control including 
screening for low counts of reads per sample. All samples were sequenced using 
the Illumina HiSeq platform (2 x 100 nucleotide sequences and an index read).

\subsection{Accession Numbers}

Raw sequences for the dog samples have been deposited to the \gls{ena} at the 
following accession number ERP014919, equivalent processed \gls{otu} tables and 
metadata can be accessed through 
Qiita\footnote{\label{qiitaurl}\url{https://qiita.microbio.me}} under study 833 
- `Dog models of inflammatory bowel disease'.

The data for the human dataset\cite{RN154} can be found in the \gls{ena} at the 
following accession numbers ERP015241 and ERP015242, equivalent processed 
\gls{otu} tables and metadata can be accessed through 
Qiita\textsuperscript{\ref{qiitaurl}} under study identifiers 1939 and 1998 
-`The Treatment-Naive Microbiome in New-Onset Crohn's Disease'.

Data for the additional dog study \cite{RN153} can be found at the \gls{sra} of 
the under accession number SRP040310.

\chapter{Sample collection and processing for 
Chapter~\ref{chapter_ibds}}\label{appendix_ibds}

\section{Methods for Chapter~\ref{section_plane}}\label{appendix_plane}

\subsection{Cohort Demographics}

Patients with \gls{cd} or \gls{uc}, the two major forms of \gls{ibd}, attending 
the outpatient clinic were consecutively invited to take part. After obtaining 
written consent, \gls{bmi} was recorded and patients were asked to provide a 
fecal sample and to fill in a questionnaire with clinical disease activity, 
present medication, dietary habits, use of antibiotics and use of 
\glspl{nsaid}. Disease phenotype was classified according to the Montreal 
classification \cite{Silverberg2005}. Individuals were then followed 
prospectively, asked to provide fecal samples and to fill in the questionnaire 
every third month for a two-year period. If a patient did not provide a fecal 
sample at any of the three months periods, a reminder letter was sent. In total 
109 patients with \gls{ibd} (\gls{cd}; n=49 and \gls{uc}; n=60) took part. Nine 
additional individuals with no \gls{ibd} or any other gastrointestinal 
conditions were recruited as \gls{hc} as well as 19 patients with other chronic 
inflammatory gastrointestinal diseases (4 \gls{lc} and 15 \gls{cc}).  All 137 
individuals were Caucasians and together they provided 683 fecal samples during 
the two-year period (Supplementary 
Table~4\footnote{\url{https://images.nature.com/original/nature-assets/nmicrobiol/2017/nmicrobiol20174/extref/nmicrobiol20174-s1.pdf}}).  
The study was approved by the Ethical Committee of the Medical Faculty, Uppsala 
University (2007/291).

\subsection{Sample Collection}

Fecal samples were self-collected in sterile plastic containers and stored at 
-80 °C until shipping on dry ice and processing.

\subsection{Fecal calprotectin} 

To assess the degree of inflammatory activity at the collection of each fecal 
sample, the concentration of f-calprotectin was assessed by commercially 
available ELISA, Calprotectin Elisa Buhlmann Laboratories AG, Basel, 
Switzerland, according to the manufacturer's protocol. 

\subsection{DNA Extraction and Amplification}

Genomic DNA was extracted from 0.25 g of fecal material from each sample using 
the Earth Microbiome DNA extraction protocol \cite{EMP2011}. Briefly, DNA was 
extracted using the 96-well format MoBio Powersoil DNA kit on an EpMotion 5075 
robot with vacuum (Eppendorf, Hamburg, Germany). DNA was quantified with the 
Qubit 2.0 fluorometer (Invitrogen, Carlsbad, CA) according to the 
manufacturer's instructions.

PCR amplification and library preparation were performed similarly to the 
protocol described by Caporaso et al. \cite{Caporaso2011proceedings}. 515F/806R 
Illumina primers with unique reverse primer barcodes were used to target the V4 
region of the 16S rRNA gene. Samples were amplified in triplicate and cleaned 
using the MO BIO 69 htp PCR cleanup kit. Each PCR reaction included 1X PCR 
buffer, 10 $\mu$M each forward and reverse primer, 200 μM dNTPs, 1 U/ml Taq 
polymerase, 15 ng template DNA, and PCR grade water, with a total reaction 
volume of 25 $\mu$L. Reactions were kept at 94\textdegree C for 3 minutes for 
denaturation to occur. Amplification was performed by 25 cycles of 
94\textdegree C for 45s, 58\textdegree C for 60s, and 72\textdegree C for 90s.  
The V4 amplicons were sequenced on the Illumina HiSeq 2000 platform, yielding 
single end, 100 base pair reads. Sequencing and quality assessment were 
performed at the Yale Center for Genome Analysis.

\subsection{Data Availability}

Microbiome data from this study is available on Qiita under study ID 1629 
(https://qiita.ucsd.edu/study/description/1629) and using the EBI accession 
number ERP020401. Patient clinical information is available on Qiita and in 
Supplemental 
Dataset~1\footnote{\url{https://images.nature.com/original/nature-assets/nmicrobiol/2017/nmicrobiol20174/extref/nmicrobiol20174-s2.txt}}. 

\section{Methods for Chapter~\ref{section_ibd}}\label{appendix_ibd}

\subsection{Fecal collection}

Stool was collected daily using a swab technique \cite{RN4220}, which enables 
the study subject to collect stool samples from the toilet paper. Samples were 
stored on a -20\textdegree C freezer. Fifteen patients collected daily stool 
for two 2-week periods separated by an interval of 4 weeks, during which no 
stool was collected.  The short \gls{cdai} was evaluated at entry and at 
the end of collection period 1 and 2 \cite{RN4006}. The collection periods 
for the family substudy, which included 4 patients with inactive \gls{cd} 
with same exclusion criteria as above and 3 unaffected family members of 
each \gls{cd} patient were two separate 4 weeks periods interrupted by a 3 
month collection-free interval.

Patients with proven \gls{cd} diagnosed for at least 3 months and in clinical 
remission were recruited at the University of North Carolina (see 
Table~\ref{ibd-table1}). Serial stool samples as outlined below were submitted 
for analysis. Exclusion criteria for entry into the study were active \gls{cd} 
as defined by a short \gls{cdai} score $>$ 150; fistulizing \gls{cd}, 
concomitant use of azathioprine (AZA), 6-mercaptopurine (6-MP), methotrexate or 
anti-TNF agents for less than 3 months; or concomitant use of systemic steroids 
or budesonide. Steroids or budesonide had to be discontinued at least 8 weeks 
before inclusion, and local, rectally administered therapies containing 5-ASA 
(enemas, suppositories) or steroid enemas/foams should have not been used for 
the previous 4 weeks. Also, \glspl{nsaid} were not allowed as regular 
treatment, which was defined as use for at least 4 days a week each month.  
Patients were excluded if they were on antibiotic therapy $\geq$5 days each 
month or had antibiotic therapy $\geq$5 days in the previous 24 weeks. No 
probiotics were allowed in the last 24 weeks before inclusion and patients on 
long-term therapy with narcotics $>$2 days weekly were excluded.  Further 
subject exclusions were known Hepatitis B, Hepatitis C or PSC or regular, high 
dose alcohol consumption (more than seven drinks per week). All trial 
participants were prohibited from consuming specific diets (e.g. Atkins diet, 
low carbohydrate diet). In the second subset we also included nonaffected 
family members to collect serial fecal samples.  The requirement for the family 
subjects was one sibling age $\geq$ 8 years without known \gls{cd} or 
ulcerative colitis and two living parents without \gls{cd} or ulcerative 
colitis. The same exclusion criteria for antibiotics, probiotics, narcotics and 
concomitant diseases as in the main study were applied. 

\begin{table}[htbp]
    \centering
    \caption{Summary of the demographic data of the 19 CD patients.}
    \label{id-tabMethods}
    \begin{tabular*}{\textwidth}{rrl}
    \toprule
         \multicolumn{2}{r}{Male/Female (n)} &	8 / 11\\
    \midrule
    \multicolumn{2}{r}{Age (years, median; range)} &	31 (15-51)\\
\midrule
\multicolumn{2}{r}{Duration of Crohn's disease (years median; range)} &	9 (0.5-35)\\
\midrule
\multirow{3}{*}{Location of Crohn's disease (n)} & 	Ileal & 5\\
 &   	Ileo-colonic & 10\\
 &     	Colonic	 & 4\\
\midrule
\multicolumn{2}{r}{History of ileocecal resection} &	7\\
\midrule
\multicolumn{2}{r}{History of isolated small intestinal resection} &	2\\
\midrule
\multirow{4}{*}{Concomitant therapies (n)} &        	Steroids & 0\\
&        	Azathioprine/6-MP/methotrexate & 8\\
 &        	anti-TNF agent & 12\\
     &   	5-ASA	 &2\\
\midrule
\multicolumn{2}{r}{\gls{cdai} end of week 2 (mean; SD)} &	63 (51)\\
\midrule
\multicolumn{2}{r}{\gls{cdai} end of week 4 (mean; SD)} &	72 (61)\\
\bottomrule
    \end{tabular*}
\end{table}

\subsection{DNA Extraction}

Fecal DNA isolation was performed according to the 16S Earth Microbiome Project 
Protocol \cite{RN164}. Of the 960 samples analyzed, the initial subset (384 
samples)  were processed using the Illumina MiSeq platform (150 nucleotide 
sequences), and for the second subset (576 samples) were processed using the 
HiSeq platform (100 nucleotide sequences).

\subsection{Data Availability}

The sequences have been deposited on EBI and are available under the following
accession number ERP104742.  In addition, the processed sequences and sample
information can be found in the
Qiita\footnote{\url{https://qiita.ucsd.edu/study/description/2538}} database
under the study identifier 2538.

\chapter{Sample collection and processing for 
Chapter~\ref{chapter_fmts}}\label{appendix_fmts}

\section{Methods for Chapter~\ref{section_moviefmt}}\label{appendix_moviefmt}

\subsection{Patients and donors}
All patients suffered from multiply \gls{rcdi} refractory to standard 
antibiotic therapies. A single standard donor was used in the preparation of 
all fecal microbiota material as described previously \cite{RN45}. The 
Institutional Review Board at the University of Minnesota approved prospective 
collection of fecal specimens and their analysis. All patients satisfied the 
inclusion criteria for the \gls{fmt} within our program, which included at 
least two spontaneous recurrences of \gls{cdi} within a month of 
discontinuation of antibiotics and failure of at least one advanced antibiotic 
regimen such as a vancomycin pulse/taper protocol or vancomycin treatment 
followed by administration of rifaximin or fidaxomicin for 2-3 weeks.  The 
specific clinical characteristics of patients involved in this study are 
summarized in Supplementary 
Table~1\footnote{\url{https://static-content.springer.com/esm/art\%3A10.1186\%2Fs40168-015-0070-0/MediaObjects/40168_2015_70_MOESM3_ESM.xlsx}}.  

\subsection{Fecal microbiota transplantation}
\Gls{fmt} was done using a standardized preparation of concentrated fresh or 
frozen fecal bacteria via colonoscopy as previously described \cite{RN45}. All 
patents were treated with oral vancomycin, 125 mg four times daily, until two 
days prior to the procedure \cite{RN45}. The day before the procedure, patients 
received a polyethylene glycol-based colonoscopy prep 
(GoLYTELY\textsuperscript{\textregistered} or 
MoviPrep\textsuperscript{\textregistered}) to remove residual antibiotics and 
fecal material. Donor fecal microbiota was placed into the terminal ileum 
and/or cecum via the biopsy channel of the colonoscope. A total of 17 donor 
samples from the same individual were used in these studies.  The CD1-CD4 donor 
samples were given to patients CD1-CD4, respectively. Patients CD1, CD3, and 
CD4 received freshly prepared fecal microbiota, while patient CD2 received a 
previously frozen preparation of fecal microbiota, all from the same 
standardized, anonymous donor.

\subsection{Sample collection}
Fecal samples were collected at home by the patients using swabs to sample 
feces deposited into a toilet hat immediately after production, and stored 
frozen at approximately -20\textdegree C. Samples were subsequently transferred 
to the laboratory and stored at -80\textdegree C until used. Donor samples for 
DNA extraction were collected during processing of material for \gls{fmt}, and 
stored frozen at       -80\textdegree C until used. Samples from patients CD1- 
4 were obtained prior to \gls{fmt} and between 1 to 151 days post-\gls{fmt}, 
with daily collection until day 28, and weekly collection until day 84. Fecal 
material prior to \gls{fmt} was obtained from patients CD5-CD14.

\subsection{DNA extraction}
DNA was extracted from donor and recipients' pre- and post-\gls{fmt} fecal 
samples using MOBIO PowerSoil DNA extraction kits (MOBIO, Carlsbad, CA), 
according to the manufacturer's instructions. Fecal DNA concentrations were 
measured using a QuBit DNA quantification system (Invitrogen, Carlsbad, CA).

\subsection{PCR amplification}
Extracted DNA was amplified using the EMP standard 
protocols\footnote{\url{http://www.earthmicrobiome.org/}} following the 
recommendations of Caporaso et al. \cite{RN4221}. Briefly, F515/R806 primers 
were used, with 12-base Golay codes introduced on the 806 end to provide unique 
sample indices. Cycling and annealing conditions were as previously described 
\cite{RN4221}.

\subsection{DNA sequencing}
DNA sequencing was performed as previously described \cite{RN4221} on an 
Illumina MiSeq platform using 2 x 150 bp paired-end reads and the Illumina v3 
reagent chemistry. 

\section{Methods for Chapter~\ref{section_fmt}}\label{appendix_fmt}

\subsection{Patient selection}

Patients undergoing \gls{fmt} for \gls{rcdi} were prospectively recruited in 
this study. Informed consent was obtained to collect clinical data and stool 
samples. Data collected included demographics, clinical history, \gls{cdi} 
treatment history, comorbid conditions and response to \gls{fmt}. A donor fecal 
sample was collected prior to \gls{fmt}. Stool samples from the recipients were 
collected before \gls{fmt}, and at day 7 and day 28 and were stored at 
-80\textdegree C. The donors were either related (genetically related family 
members) or unrelated (screened hospital employee volunteer donors or unrelated 
family members) and a fresh sample was obtained on the day of \gls{fmt}. All 
donors underwent extensive screening in accordance with standard practice and 
guidelines from the Food and Drug Administration \cite{RN1446}. Donor selection 
criteria and experience from our group have been previously published 
\cite{RN1523}. The donor stool sample is weighed and divided into 50 grams 
aliquots. Each aliquot of 50 grams is diluted in normal saline in a 1:5 ratio 
(50 grams of stool diluted with 250 ml of normal saline) and is placed in the 
blender bag (a 2 bag system with a semipermeable membrane in the inner bag and 
the outside bag is plastic). The stool is placed in the inner bag and normal 
saline is added. The bag is placed in a sealed compartment in the stomacher 400 
(Seward) and blended for 60 seconds at 230 rotations per minute. The filtrate 
is then placed into 50 ml conical tubes using 50 ml pipettes and placed on an 
ice pack prior to the procedure.  \Gls{rcdi} was defined as another episode of 
\gls{cdi} within 56 days after symptom resolution with recurrence of symptoms 
and a positive stool polymerase chain reaction test. For this study, future 
\textit{C. difficile} episodes after \gls{fmt} up to 2 years were captured.  
These were categorized as up to 56 days, 56 days to 1 year and beyond 1 year. 

\subsection{Sequencing and analytic methods}

After fecal DNA isolation (MoBio, Carlsbad, CA fecal DNA kit), amplicons 
spanning the variable region 4 of bacterial 16S rRNA were generated and 
sequenced using Illumina MiSeq platform at the Mayo Clinic Medical Genome 
Facility, Rochester, MN. The 16S rRNA sequencing data from the Illumina runs 
were quality controlled, trimmed, demultiplexed and assigned to \glspl{otu} 
following the closed reference at 97\% similarity (using SortMeRNA as a 
clustering algorithm \cite{RN3810} protocol against the Greengenes \cite{RN165} 
database 13\textunderscore 8 release, as implemented in \gls{qiime} 1.9.0 
software \cite{RN110}, default parameters were used for all these steps unless 
otherwise noted. After quality control, 10,583,052 sequences were obtained, for 
a mean of 76,688 sequences per sample (min: 33,559, max: 154,200).


\subsection{Clinical Statistics}

Statistical analyses for clinical data were performed with JMP version 11.0 
(SAS institute, NC).  Data analysis included descriptive statistics, t-tests 
for normally distributed variables, non-parametric tests for skewed variables, 
chi-square tests and ANOVA tests as applicable. A p-value less than 0.05 was 
considered statistically significant.
